\documentclass[12pt]{article}

\usepackage{amsmath}
\usepackage[portuguese]{babel}
\usepackage{graphicx}
\usepackage{float}
\usepackage{fancyhdr}
\usepackage{geometry}
  \geometry{
    a4paper,
    vmargin={25mm,25mm},
    hmargin={25mm,25mm},
    includehead,
    head={15.5mm}
  }
\renewcommand{\figurename}{Figura}
\renewcommand{\tablename}{Tabela}
\setlength{\parindent}{1.5cm}
\usepackage{indentfirst}
\usepackage[indent=15mm]{parskip}
\pagestyle{fancy}
\fancyhead{}
\fancyhead[r]{\includegraphics[width=25mm]{imagens/DIN_logo.png}}
\fancyhead[C]{\textbf{UNIVERSIDADE ESTADUAL DE MARINGÁ\\CENTRO DE TECNOLOGIA\\DEPARTAMENTO DE INFORMÁTICA}}
\fancyhead[l]{\includegraphics[width=25mm]{imagens/UEM_logo.png}}
\renewcommand{\contentsname}{\centering SUMÁRIO}

\begin{document}

\begin{titlepage}
\begin{center}
       {CIÊNCIA DA COMPUTAÇÃO}

       \vspace{2cm}
       
       \textbf{Programação para Interfaceamento de Hardware e Software (9792)}
  
       \vspace{1cm}

       \textbf{TRABALHO 2}

       \vfill
       
       \textbf{Professor: Ronaldo Augusto de Lara Gonçalves}
       
       \vspace{1cm}

       \textbf{Discentes}
       \begin{table}[h]
           \centering
           \begin{tabular}{|c|c|}\hline
                RA & NOME \\ \hline
                130099 & GUSTAVO HENRIQUE TRASSI GANAZA\\ \hline
                129182 & YOSHIYUKI FUGIE \\ \hline
            \end{tabular}
       \end{table}

            Maringá
            
            2025
\end{center}
\end{titlepage}

\tableofcontents
\newpage

\section{Visão Geral do Sistema}
Este relatório detalha a implementação do sistema de gerenciamento de produtos em Assembly 32 bits. O sistema utiliza uma lista encadeada para armazenar os produtos, com operações para adicionar, buscar, remover, atualizar, gerar relatórios e realizar consultas financeiras. O código é organizado em funções que manipulam a lista e interagem com o usuário via terminal.

A entrega inclui um arquivo makefile para compilar o código-fonte. O makefile utiliza \textit{gcc -m32} para gerar código 32 bits, portanto é necessário ter instalado o pacote \textit{gcc-multilib} no sistema (em distribuições baseadas no Debian/Ubuntu: \textit{sudo apt install gcc-multilib}). Para compilar o código, execute o comando \textit{make} no terminal. Para rodar o programa, execute \textit{./supermercado}.

\section{Estrutura de Dados}
A estrutura de um produto é definida com os seguintes campos:
\begin{itemize}
    \item \textbf{next}: ponteiro para o próximo nó (4 bytes)
    \item \textbf{tipo}: inteiro (4 bytes)
    \item \textbf{quantidade}: inteiro (4 bytes)
    \item \textbf{valor\_compra}: inteiro (em centavos, 4 bytes)
    \item \textbf{valor\_venda}: inteiro (em centavos, 4 bytes)
    \item \textbf{nome}: string (50 bytes)
    \item \textbf{lote}: string (20 bytes)
    \item \textbf{dia}: inteiro (4 bytes)
    \item \textbf{mês}: inteiro (4 bytes)
    \item \textbf{ano}: inteiro (4 bytes)
    \item \textbf{fornecedor}: string (50 bytes)
\end{itemize}

O tamanho total da estrutura é 152 bytes, sendo 4 bytes para o ponteiro e 148 bytes para os dados.

A lista é acessada através de um ponteiro global \texttt{head} que aponta para o primeiro nó.

\section{Registradores principais}
    \begin{description}
      \item[EAX] Acumulador / retorno de função / aritmética.
      \item[EBX] Preservado por chamadas (\emph{callee-saved}), guarda ponteiros de nó, handle de arquivo, buffer, etc.
      \item[ECX] Contador em loops, ou divisor em \texttt{divl}.
      \item[EDX] Resto de divisão / auxiliares em formatação de moeda.
      \item[ESI, EDI] Índices / preservados, usados para percorrer listas ou passar handles.
      \item[EBP, ESP] Gerenciam o frame de pilha de cada função.
    \end{description}

\section{Adição de Produto}
Função: \texttt{add\_product\_interactive}
Esta função interage com o usuário para coletar os dados do produto, aloca memória para o novo nó e o insere na lista de forma ordenada pelo nome.
\textbf{Registradores principais:}
\begin{itemize}
    \item \textbf{EAX}: usado para armazenar o endereço do novo nó alocado.
    \item \textbf{EBX}: mantém o endereço do novo nó durante a coleta de dados.
\end{itemize}
\textbf{Fluxo:}
\begin{enumerate}
    \item Aloca memória para o novo produto (152 bytes) usando \texttt{malloc}.
    \item Inicializa o ponteiro \texttt{next} do novo nó como NULL.
    \item Lê os campos do produto (nome, lote, tipo, dia, mês, ano, fornecedor, quantidade, valor de compra, valor de venda) usando funções auxiliares (\texttt{read\_string\_with\_prompt}, \\ \texttt{printf}, \texttt{scanf}).
    \item Chama \texttt{insert\_sorted} para inserir o nó na lista mantendo a ordem alfabética pelo nome.
\end{enumerate}
\textbf{Função \texttt{insert\_sorted}:}
\begin{itemize}
    \item \textbf{EBX}: contém o endereço do novo nó.
    \item \textbf{EDI}: ponteiro atual na lista (começa em \texttt{head}).
    \item \textbf{ESI}: ponteiro anterior (para inserção).
    \item Compara o nome do novo nó com os nomes dos nós existentes usando \texttt{strcmp}.
    \item Insere o nó na posição correta:
        \begin{itemize}
            \item Se a lista estiver vazia, \texttt{head} aponta para o novo nó.
            \item Se o novo nó deve ser o primeiro, atualiza \texttt{head} e o ponteiro \texttt{next} do novo nó.
            \item Caso contrário, insere no meio ou no final.
        \end{itemize}
\end{itemize}

\section{Busca de Produto}
Função: \texttt{search\_product\_interactive}

Esta função lê um nome do usuário e percorre a lista imprimindo todos os produtos com nomes que coincidem.

\textbf{Registradores principais:}
\begin{itemize}
    \item \textbf{EBX}: ponteiro para o nó atual durante a busca.
    \item \textbf{EAX}: usado para chamadas de função e comparações.
\end{itemize}

\textbf{Fluxo:}
\begin{enumerate}
    \item Lê o nome a ser buscado.
    \item Chama \texttt{search\_product} com o nome.
    \item Em \texttt{search\_product}:
        \begin{enumerate}
            \item Inicia no \texttt{head}.
            \item Para cada nó, compara o nome com o nome buscado usando \texttt{strcmp}.
            \item Se coincidir, imprime o produto usando \texttt{print\_product}.
            \item Avança para o próximo nó.
        \end{enumerate}
\end{enumerate}


\section{Remoção de Produto}
Função: \texttt{remove\_product\_interactive}

Remove um produto específico baseado no nome e lote.

\textbf{Registradores principais:}
\begin{itemize}
    \item \textbf{EBX}: ponteiro para o nó atual.
    \item \textbf{ESI}: ponteiro para o nó anterior (para ajustar os ponteiros).
\end{itemize}

\textbf{Fluxo:}
\begin{enumerate}
    \item Lê o nome e o lote do produto a ser removido.
    \item Percorre a lista:
        \begin{enumerate}
            \item Compara o nome e o lote do nó atual com os valores lidos.
            \item Quando encontra, remove o nó:
                \begin{itemize}
                    \item Se for o primeiro nó, atualiza \texttt{head}.
                    \item Se estiver no meio, o nó anterior (\texttt{ESI}) aponta para o próximo do nó removido.
                \end{itemize}
            \item Libera a memória do nó com \texttt{free}.
        \end{enumerate}
\end{enumerate}


\section{Atualização de Produto}
A função \texttt{update\_product\_interactive}:
\begin{enumerate}
  \item Lê nome e lote para localizar o nó.
  \item Pergunta campo a atualizar (quantidade ou valor de venda).
  \item Usa \texttt{scanf} + \texttt{clear\_input\_buffer} para ler o inteiro e grava no offset correspondente (8 para quantidade, 16 para venda).
  \item Confirma sucesso ou falha via mensagens.
\end{enumerate}

\section{Consultas Financeiras}
Função: \texttt{finance\_menu}
Apresenta um submenu para o usuário escolher entre:
\begin{enumerate}
    \item Total gasto em compras.
    \item Total estimado de vendas.
    \item Lucro total estimado.
    \item Capital perdido (produtos vencidos).
\end{enumerate}

\textbf{Fluxo:}
\begin{itemize}
    \item \texttt{total\_compra}: Percorre a lista somando \texttt{quantidade * valor\_compra} para cada produto.
    \item \texttt{total\_venda}: Percorre a lista somando \texttt{quantidade * valor\_venda} para cada produto.
    \item \texttt{lucro\_total}: Calcula \texttt{total\_venda - total\_compra}.
    \item \texttt{capital\_perdido}:
        \begin{enumerate}
            \item Pede a data atual ao usuário.
            \item Para cada produto, compara a data de validade com a data atual usando \texttt{compare\_dates}.
            \item Se a validade for anterior à data atual, soma \texttt{quantidade * valor\_compra} ao total perdido.
        \end{enumerate}
    \item Os resultados são impressos em formato de moeda (reais.centavos).
\end{itemize}

\textbf{Registradores nas funções de soma:}
\begin{itemize}
    \item \textbf{EBX}: ponteiro para o nó atual.
    \item \textbf{ESI}: acumulador do total.
\end{itemize}

\section{Persistência em Disco}
\subsection*{Gravação (\texttt{save\_list})}
\begin{enumerate}
  \item Abre \texttt{produtos.bin} em modo \texttt{wb}.
  \item Para cada nó: copia \texttt{dados\_size} bytes (offset 4) para um buffer via \texttt{memcpy}, chama \texttt{fwrite}.
  \item Fecha arquivo (\texttt{fclose}).
\end{enumerate}

\subsection*{Carregamento (\texttt{load\_list})}
\begin{enumerate}
  \item Abre \texttt{produtos.bin} em modo \texttt{rb}.
  \item Enquanto \texttt{fread} retornar \texttt{dados\_size}: aloca nó (\texttt{malloc}), copia buffer para \texttt{novo+4}, chama \texttt{insert\_sorted}.
  \item Fecha arquivo.
\end{enumerate}

\section{Geração de Relatórios}
Função: \texttt{generate\_report}

Gera um relatório em arquivo texto (\texttt{relatorio.txt}) com todos os produtos. O usuário pode escolher a ordenação: por nome (padrão), por quantidade ou por data de validade (mais antiga primeiro).

\textbf{Fluxo:}
\begin{enumerate}
    \item Abre o arquivo para escrita.
    \item Pergunta ao usuário o critério de ordenação.
    \item Dependendo da escolha:
        \begin{itemize}
            \item \textbf{Ordenação por nome}: percorre a lista normalmente (já está ordenada por nome).
            \item \textbf{Ordenação por quantidade ou data}: 
                \begin{enumerate}
                    \item Conta o número de nós.
                    \item Aloca um array de ponteiros para nós.
                    \item Preenche o array com os ponteiros dos nós.
                    \item Ordena o array (usando \texttt{sort\_by\_quantity} ou \texttt{sort\_by\_date}).
                    \item Percorre o array ordenado, escrevendo cada produto no arquivo.
                    \item Libera o array.
                \end{enumerate}
        \end{itemize}
    \item Para cada produto, escreve todos os campos no arquivo usando \texttt{fprintf}.
    \item Fecha o arquivo.
\end{enumerate}

\textbf{Ordenação:}
\begin{itemize}
    \item \texttt{sort\_by\_quantity}: Usa bubble sort no array de ponteiros, comparando o campo \texttt{quantidade}.
    \item \texttt{sort\_by\_date}: Usa bubble sort comparando as datas. A função de comparação \texttt{compare\_nodes\_by\_date} extrai dia, mês e ano de cada nó e compara (ano, depois mês, depois dia).
\end{itemize}
\textbf{Registradores nas funções de ordenação:}
\begin{itemize}
    \item \textbf{ESI}: endereço do array.
    \item \textbf{ECX}: número de elementos.
    \item \textbf{EDI}, \textbf{EBX}: índices para loops.
    \item \textbf{EAX}, \textbf{ECX}: ponteiros para nós durante a comparação.
\end{itemize}

\section{Limitações e Pontos de Melhoria}
Embora funcional, o código ainda apresenta algumas fragilidades:
\begin{itemize}
  \item \textbf{Validação de entrada ausente:} todas as chamadas a \texttt{scanf} ou \texttt{read\_int} assumem que o usuário fornecerá dados válidos. É possível, por exemplo, ler dia, mês ou ano negativos, ou valores fora de faixa.
  \item \textbf{Sem tratamento de erro em tempo de execução:} não há verificação do retorno de \texttt{fread}, \texttt{fwrite} ou \texttt{malloc} além do teste simples de ponteiro nulo. Em caso de falha parcial de I/O, o relatório ou a lista podem ficar corrompidos.
  \item \textbf{Possíveis vazamentos de memória:} em \texttt{generate\_report}, se a alocação de \texttt{node\_array} falhar, a função retorna sem liberar memória prévia. Similarmente, em rotinas CRUD, \texttt{malloc} bem-sucedido que não é seguido de \texttt{insert\_sorted} em caso de erro não é sempre liberado.
  \item \textbf{Limite fixo de buffers:} funções como \texttt{read\_string\_with\_prompt} usam \\ \texttt{fgets(buffer,50)} sem checar se o usuário excedeu 49 caracteres, o que pode resultar em truncamento silencioso.
  \item \textbf{Overflow aritmético:} ao multiplicar quantidade por valor em centavos, não há verificação de estouro de 32 bits em \texttt{imull}, podendo levar a resultados incorretos em consultas financeiras.
\end{itemize}

\end{document}